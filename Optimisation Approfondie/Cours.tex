\documentclass{article}
\usepackage[utf8]{inputenc}
\usepackage[a4paper, margin=2.5cm]{geometry}
\usepackage{graphicx}
\usepackage[french]{babel}

\usepackage[default,scale=0.95]{opensans}
\usepackage[T1]{fontenc}
\usepackage{amssymb} %math
\usepackage{amsmath}
\usepackage{amsthm}
\usepackage{bbm} % Pour l'indicatrice
\usepackage{systeme}

% pour les hyperlien 
\usepackage{hyperref}
\hypersetup{
    colorlinks=true,
    linkcolor=blue,
    filecolor=magenta,      
    urlcolor=cyan,
    pdftitle={Overleaf Example},
}
\urlstyle{same} % utiliser \href{url}{Text}

\theoremstyle{plain}% default
\newtheorem{thm}{Théorème}[section]
% \newtheorem{lem}[thm]{Lemme}
% \newtheorem{prop}[thm]{Proposition}
% \newtheorem*{cor}{Corollaire}
% %\newtheorem*{KL}{Klein’s Lemma}

\theoremstyle{definition}
\newtheorem{defn}{Définition}[section]
\newtheorem{exmp}{Exemple}[section]
% % \newtheorem{xca}[exmp]{Exercise}

% \theoremstyle{remark}
\newtheorem*{rem}{Remarque}
\newtheorem*{note}{Note}
% %\newtheorem{case}{Case}



\title{Cours}
\author{Charles Vin}
\date{Date}

\begin{document}
\maketitle


\section{Quelques rappels et notation}
\subsection{Matrice symétrique réelle}
\begin{defn}[Définie positive]
    $ A $ est DP si 
    \[
        \forall x \in \mathbb{R}\setminus \{0\} : x^T A x > 0
    .\]
    Plus concrètement on diagonalise et regarde le signe des valeurs propres.
\end{defn}

\begin{thm}[décomposition de Schan]
    Soit $ A \in S_n(\mathbb{R}) $ alors il existe $ U $ matrice unitaire ($ U^TU=I \Leftrightarrow U^{-1}=U^T $ ) et $ D $ matrice diagonale tq 
    \[
        A=U^TDU
    .\]
    $ D=diag(\lambda _1, \dots, \lambda _n) $ 
    $ U={u_1,\dots,u_2} $ 
    \begin{note}[]
        C'est le théorème de la diagonalisation ça !
    \end{note}
\end{thm}
\begin{rem}[]
    Comme $ U $ est inversible on a 
    \[
        \forall y \in \mathbb{R}^n, \exists x \in \mathbb{R}^n, y=U^Tx
    .\]
\end{rem}

\begin{defn}[Dérivé directionnel]
    Permet les dérivés de matrice.
    \[
        \nabla f(x) * h = \lim_{t \to 0} \frac{f(x+th) - f(x)}{t}
    .\]
\end{defn}

\begin{defn}[Formule de Lagrange]
    
\end{defn}

\subsubsection{Exercise 1 page 3:}
Voir OneNote

\begin{table}[!ht]
    \centering
    \begin{tabular}{|l|l|l|l|}
    \hline
        $f$ & $g$ & $F$ & $\nabla R$ \\ \hline
        $\mathbb{R}^n \to \mathbb{R}^m $ & $\mathbb{R}^n \to \mathbb{R}^m $ & $ f \pm g, f^Tg \in \mathbb{R}$ & $ \nabla f \pm \nabla g , g^T \nabla f + f^T \nabla g \in \mathbb{R}^{1\times n} $ \\ \hline
        $\mathbb{R}^n \to \mathbb{R} $ & $\mathbb{R}^n \to \mathbb{R}^m $ & $ fg \in \mathbb{R}^m $ & $g \nabla f + f \nabla g \in \mathcal{M}_{m,n}(\mathbb{R}) $ \\ \hline
        $\mathbb{R}^n \to \mathbb{R}^m $ & $\mathbb{R}^n \to \mathbb{R}^p $ & $ g \circ f \in \mathbb{R}^p $ & $ \nabla g(f) \nabla f \in \mathcal{M}_{p,n}(\mathbb{R}) $ \\ \hline
    \end{tabular}
\end{table}

\begin{defn}[Convexité]
    Un ensemble $ D \subset \mathbb{R}^n $ est un \textbf{convexe} si 
    \[
        \forall x,y \in D, \forall t \in [0,1], (tx + (1-t)y) \in D
    .\]
\end{defn}

\begin{defn}[Fonction convexe]
    Un fonction $ f $ est (strictement ou non) convexe sur $ D $ si 
    \[
        \forall x,y \in D, \forall t \in [0,1], f(tx+(1-t)y) \leq t f(x) + (1-t)f(y)
    .\]
\end{defn}

\begin{thm}[]
    Soit $ f:D \to \mathbb{R}^n $ une fonction de classe $ \mathcal{C}^2 $ alors 
    \begin{enumerate}
        \item $ f $ est convexe $ \Leftrightarrow \forall x \in D: \nabla ^2 f(x) est SDP$ 
        \item $ \forall x \in D: \nabla ^2 f(x) $ est DP $ \Rightarrow f $ est strictement convexe
    \end{enumerate}
\end{thm}

\begin{table}[!h]
    \centering
    \begin{tabular}{|l|l|l|}
    \hline
        $ f(x) $                    & $ \nabla f(x) $                           & $ \nabla ^2 f(x) $            \\ \hline
        $ c \in \mathbb{R}^N $      & $ 0 \in \mathcal{M}_{n,m}(\mathbb{R}) $   & $ ~ $                         \\ \hline
        $ b^T x \in \mathbb{R} $    & $ b^T \in \mathbb{R}^{1 \times n} $       & $ 0 \in S_n(\mathbb{R}) $     \\ \hline
        $ bx \in \mathbb{R}^n $     & $ B*I \in S_n(\mathbb{R}) $               & $ 0 \in S_n(\mathbb{R}) $     \\ \hline
        $ Ax \in \mathbb{R}^m $     & $ A \in \mathcal{M}_{m,n}(\mathbb{R}) $   & $ 0 \in S_n(\mathbb{R}) $     \\ \hline
        $ x^Tx $                    & $ 2x^T $                                  & $ 2 I \in S_n(\mathbb{R}) $   \\ \hline
        $ x^T A x $                 & $ x^t (A + A^T) $                         & $ A+A^T \in S_n(\mathbb{R}) $ \\ \hline
    \end{tabular}
\end{table}

\begin{thm}[Formule de lagrange]
    Soit 
    \begin{itemize}
        \item $ (P) $ un problème d'optimisation à $ p $ contraintes d'égalités et $ q $ contraintes d'inégalités
        \item $ g_i (x) = 0 $ les contraintes d'égalités 
        \item $ h_j(x) \leq 0 $ les contraintes d'inégalités
    \end{itemize}
    \[
        L(x, \lambda , \mu ) = f(x) - \sum_{i=1}^{p}\lambda _i g_i(x) - \sum_{j=1}^{q}\mu _i h_j(x)
    .\]
    On cherche ensuite la borne inférieur de la fonction dual de lagrange (en dérivant)
    \[
        D(\lambda , \mu ) = \inf _{x \in \mathcal{A}} L(x, \lambda , \mu )
    .\]
\end{thm}


\section{Convexité}
\begin{defn}[]
    $ f : E \to F$ convexe ssi 
    \[
        \forall t \in [0,1], x,y \in E : f(tx + (1-t)y) \leq tf(x) + (1-t)f(y)
    .\]
\end{defn}
\begin{thm}[1.a]
    $ f \text{ convexe } \Leftrightarrow $ \begin{itemize}
        \item $\nabla ^2f(x) \text{ SDP }$ 
        \item Strictement convexe $ \Leftrightarrow $ au plus une solution opti.
    \end{itemize}
\end{thm}

\begin{defn}[Problème convexe]
    $ \Leftrightarrow f: E \to F $ strictement convexe et $ E $ convexe. Souvent $ E=\mathbb{R}^n $ qui est convexe. \\
    propriété (thm 2) : \begin{itemize}
        \item Tout minimum local est un min globale 
    \end{itemize}
\end{defn}


\section{Optimisation sans contraintes}
\begin{thm}[Condition Nécessaires d'optimalité]
    
\end{thm}

\begin{thm}[Condition Suffisantes d'optimalité]
    
\end{thm}

\begin{exmp}[Exemple 1,2,3 page 6-7]
    
\end{exmp}

\begin{defn}[Inverse d'une matrice 2x2]
    \[
        A = \begin{pmatrix}
            a & b \\
            c & d
        \end{pmatrix}, A^{-1} = \frac{1}{\det A} \begin{pmatrix}
            d & -b \\
            -c & a
        \end{pmatrix}
    .\]
\end{defn}

\underline{Nouveau cours du 07/02} \\

Soit 
\begin{align*}
    (P) &min f(x) \\
    S.C & \begin{cases}
        g(x) = 0   \\
        h(x)\leq 0 \\
        X \in \Omega 
    \end{cases}
\end{align*}

\subsection{Qualification des contraintes}
\begin{itemize}
    \item On dit que les contraintes sont qualifiées en $ X \in \mathcal{A} $ si (au moins) un critère parmis les deux critères (Fromovitz ou Fiacco) est vrai en $ X $ 
    \item On dit que les contraintes sont qualifiées en tout point admissible si (au moins) un critère parmis les deux critères (affinité ou Slaten) est vrai.
\end{itemize}

\subsection{CNO}
\begin{thm}[]
    Si $ X^* $ est une solution optimalité de $ (P) $  et si les contraintes sont qualifiées en $ X^* $, alors $ X^* $ est solution du système KKT 
    \[
        \begin{cases}
        3 \text{ condition KKT}\\
        X \in \mathcal{A}\\
        \end{cases} 
    .\]
\end{thm}

\subsubsection{Résoudre un problème d'optimisation avec contraintes}
\begin{enumerate}
    \item Déterminer les points admissible où les contraintes \textbf{ne sont pas} qualifiées 
    \item Déterminer les points admissibles qui vérifient les 3 conditions KKT
\end{enumerate}
\begin{thm}[numéros 6]
    Cas particulier : $ (P) $ problème convexe
\end{thm}


\end{document}