\documentclass{article}
\usepackage[utf8]{inputenc}
\usepackage[a4paper, margin=2.5cm]{geometry}
\usepackage{graphicx}
\usepackage[french]{babel}

\usepackage[default,scale=0.95]{opensans}
\usepackage[T1]{fontenc}
\usepackage{amssymb} %math
\usepackage{amsmath}
\usepackage{amsthm}
\usepackage{systeme}
\usepackage{cases}

\usepackage{hyperref}
\hypersetup{
    colorlinks=true,
    linkcolor=blue,
    filecolor=magenta,      
    urlcolor=cyan,
    pdftitle={Overleaf Example},
    % pdfpagemode=FullScreen,
    }
\urlstyle{same} %\href{url}{Text}

\theoremstyle{plain}% default
\newtheorem{thm}{Théorème}[section]
\newtheorem{lem}[thm]{Lemme}
\newtheorem{prop}[thm]{Proposition}
\newtheorem*{cor}{Corollaire}
%\newtheorem*{KL}{Klein’s Lemma}

\theoremstyle{definition}
\newtheorem{defn}{Définition}[section]
\newtheorem{exmp}{Exemple}[section]
% \newtheorem{xca}[exmp]{Exercise}

\theoremstyle{remark}
\newtheorem*{rem}{Remarque}
\newtheorem*{note}{Note}
%\newtheorem{case}{Case}



\title{Cours}
\author{Charles Vin}
\date{Date}

\begin{document}
\maketitle

\subsection{Modéliser un problème}
\begin{enumerate}
    \item Les variables qui interviennent
    \item La fonction objectif
    \item Les contraintes
    \item Résumer le tout correctement sous un problème d'optimisation (P)
\end{enumerate}

\subsection{Résoudre un problème}
Pour un problème sous forme standard on a 
\begin{align*}
    A &= \text{ la matrice du système de contraintes} \\ 
    B &= \text{ la matrice des constantes à droite des contraintes} \\
    f^T &= \text{ matrice de la fonction à optimiser}
\end{align*}

\begin{enumerate}
    \item Résolution graphique ?
    \item Max de sommet : $ \binom{q}{n-p} $ avec $ n $ nombre de variable, $ p $ nombre de contraintes d'égalité, $ q $ nombre de contraintes d'inégalités.
    \item Mettre sous forme standard 
    \item Trouver toutes les bases possibles
    \item Pour chaque base : \begin{itemize}
        \item Vérifier si c'est réalisable : $ \det A \neq 0 \Leftrightarrow \exists A^{-1} $ puis il faut que $ X(B_i) = A^{-1}b > 0 $
        \item dégénéré ou non dégénéré : si une des variables de la base $ =0 $ (invalide la résolution du système plus tard)
        \item Correspond au sommet : $ (0,0, [X(B_i)], 0, 0, \dots) $ 
    \end{itemize}
    \item Trouver la solution optimale parmi les bases réalisables : Il faut que 
    \[
        C^{H_i} = f^T_{H_i} - f^T_{B_i} (A^{B_i})^{-1} A^{H_i} \geq 0
    .\]
\end{enumerate}

\subsubsection{Autre méthode pas encore trop compris}
\begin{itemize}
    \item Trouver $ X(B_i) $ avec
    \begin{align*}
        AX(B_i) = b &\Leftrightarrow \begin{pmatrix} A^{B_i} & A^{H_i} \end{pmatrix} \begin{pmatrix} X_{B_i} \\ X_{H_i} \end{pmatrix} = b \\
        &\Leftrightarrow X_{b_i}(\theta ) = (A^{B_2})^{-1}b - (A^{B_i})^{-1} A^{H_i} X_{H_i}(\theta )
    \end{align*}
    \item On se retrouve avec un vecteur dépendant de $ \theta  $, il faut trouver les bornes de $ \theta $ tel que \begin{itemize}
        \item $ X_{B_i} \leq 0 $
        \item $ X_{H_i} \leq 0 $
    \end{itemize}
    \item Conclure sur les points admissibles 
\end{itemize}

\end{document}