\documentclass{article}
\usepackage[utf8]{inputenc}
\usepackage[a4paper, margin=2.5cm]{geometry}
\usepackage{graphicx}
\usepackage[french]{babel}

\usepackage[default,scale=0.95]{opensans}
\usepackage[T1]{fontenc}
\usepackage{amssymb} %math
\usepackage{amsmath}
\usepackage{amsthm}
\usepackage{systeme}
\usepackage{bbm}
\usepackage{hyperref}
\hypersetup{
    colorlinks=true,
    linkcolor=blue,
    filecolor=magenta,      
    urlcolor=cyan,
    pdftitle={Overleaf Example},
    % pdfpagemode=FullScreen,
    }
\urlstyle{same} %\href{url}{Text}

\theoremstyle{plain}% default
\newtheorem{thm}{Théorème}[section]
\newtheorem{lem}[thm]{Lemme}
\newtheorem{prop}[thm]{Proposition}
\newtheorem*{cor}{Corollaire}
%\newtheorem*{KL}{Klein’s Lemma}

\theoremstyle{definition}
\newtheorem{defn}{Définition}[section]
\newtheorem{exmp}{Exemple}[section]
% \newtheorem{xca}[exmp]{Exercise}

\theoremstyle{remark}
\newtheorem*{rem}{Remarque}
\newtheorem*{note}{Note}
%\newtheorem{case}{Case}



\title{Liste des Tests du cours}
\author{Charles Vin}
\date{2022}

\begin{document}
\maketitle
\tableofcontents

\section{Template}
\subsection*{Conditions}
\subsection*{Hypothèse}
\subsection*{Statistique de test}
\subsection*{Zone de Rejet}
\subsection*{Méthode}

\section{Test de Kolmogorov-Smirnov}


\subsection*{Conditions}
\begin{enumerate}
    \item Les $ X_i $ semblent provenir d'une loi à fonction de répartition continue. $ \Rightarrow  $ on n'a pas plusieurs fois la même valeur (sauf si celle-ci on était arrondi).\\
    \item Fonctionne $ \forall n $ : même si $ n $ est petit, ce test est pertinent
    \item Si $ n \geq 100 $, on fait un test asymptotique.
\end{enumerate}

\subsection*{Hypothèse}
\begin{itemize}
    \item $ H_0 = $ les $ X_i $ ont pour fdr. $ F_X $ 
    \item $ H_1 = $ les $ X_i $ n'ont pas pour fdr. $ F_X $ 
\end{itemize}

\subsection*{Statistique de test} 
\begin{align*}
    h(F_n, F) &= \sup _{t \in \mathbb{R}} \left| F_n(t) - F(t) \right| \\
        &= \max _{1 \leq i \leq n} ( \max ( \left| \frac{i}{n} - F(X_{(i)}) \right| , \left| \frac{i-1}{n}- F(X_{(i)}) \right|  ))
\end{align*}    

\subsection*{Zone de Rejet}
\subsubsection*{Si n est petit}
La loi de $ h(F_n, F) $ est tabulé alors :
\[
    \mathcal{R} = \{h(F_n, F_X) \geq h_{1-\alpha }\}
.\]
avec $ F_n $ fonction de réparation empirique, $ h_{1 -\alpha } $ le quantile à aller chercher dans la table 

\subsubsection*{Si n est grand $ n \geq 30 $ }
Attention pas souvenir de l'avoir fait en TD. \\
On a pas la table de $ h(F_n, F) $ mais on sait que 
\[
    \sqrt[]{n}h_n \to ^{\mathcal{L}}_{n \to \infty } W_{\infty }
.\]
Donc on pose la zone de rejet 
\[
    \mathcal{R} = \{h(F_n, F_X) \geq \frac{k_\alpha }{\sqrt[]{n}} \}
.\]
avec $ F_n $ fonction de réparation empirique, $ k_{\alpha } $ le quantile de $ W_\infty  $ à aller chercher dans sa table 

\subsection*{Méthode}
Pour trouver la valeur de $ h(F_n, F_X) $ : Faire le grand tableau puis trouver le max. Exemple : 
\begin{table}[!h]
    \centering
    \begin{tabular}{|l|l|l|l|l|l|}
    \hline
        i & 1 & 2 & 3 & 4 & 5 \\ \hline
        $X_{(i)}$ & 0.3 & 0.7 & 0.9 & 1.2 & 1.4 \\ \hline
        $X_{(i)} - 2$ & -1.70 & -1.30 & -1.10 & -0.80 & -0.60 \\ \hline
        $F_0(X_{(i)})$ & 0.04 & 0.10 & 0.14 & 0.21 & 0.27 \\ \hline
        $\frac{i}{n}$ & 0.05 & 0.1 & 0.15 & 0.2 & 0.25 \\ \hline
        $|\frac{i}{n} - F_0(X_{(i)})|$ & 0.01 & 0.00 & 0.01 & 0.01 & 0.02 \\ \hline
        $|\frac{i-1}{n} - F_0(X_{(i)})|$ & 0.04 & 0.05 & 0.04 & 0.06 & 0.07 \\ \hline
    \end{tabular}
    \caption{Ici le max c'est $0.07$ à la dernière case}
\end{table}



\section{Le test du $ \mathcal{X}^2 $ d'ajustement}
\subsection*{Conditions}
\begin{enumerate}
    \item Les $ X_i $ sont à valeur dans un ensemble fini (loi discrète). Si a valeur dans $ \mathbb{N} $, on fusionne les classes à partir d'un certain rang choisis 
    \item Test asymptotique : $ \forall k \in \{1, \dots, d\}, np_k^{ref}(1-p_k^{ref}) \geq 5 \Leftrightarrow n \geq 20$ 
\end{enumerate}
Si on ne remplis pas les conditions, on peut fusionner les classes 

\subsection*{Hypothèse}
\begin{align*}
    H_0 &= p = p^{ref} \text{ i.e. } \forall k \in \{1,\dots,d\}, p_k = p_k^{ref} \\
    H_1 &= p \neq p^{ref} \text{ i.e. } \exists k \in \{1, \dots, d\}: p_k \neq p_k^{ref}
\end{align*}
Avec $ p^{ref} $ un vecteur fixé à tester (par exemple pour un lancé de dé $ (\frac{1}{6}, \dots, \frac{1}{6}) $ )

\subsection*{Statistique de test}
\begin{align*}
    D(\bar{p_n}, p^{ref}) &= n \sum_{k=1}^{d}\frac{(\bar{p_{k,n}} - p_k^{ref})^2}{p_k^{ref}} \to ^{\mathcal{L}}_{n \to \infty } \mathcal{X}^2(d-1) \\
        &= \sum_{k=1}^{d} \frac{(N_{k,n} - np_k^{ref})^2}{n p_k^{ref}}
\end{align*}
avec \begin{itemize}
    \item $ N_{k,n} = \sum_{i=1}^{n}\mathbbm{1}_{X_i x_k} $ (ce qu'il y a dans le tableau de la consigne)
    \item $ \bar{p_{k,n}} = \frac{N_{k,n}}{n} $ les proportions observés
\end{itemize}

\subsection*{Zone de Rejet}
\[
    \mathcal{R} = \{D(\bar{p_n}, p^{ref}) \geq h_\alpha\}
.\]
avec $ h_\alpha  $ le quantile d'ordre $ 1-\alpha  $ de la loi $ \mathcal{X}^2(d-1) $

\subsection*{Méthode}
\begin{enumerate}
    \item Etape 0 : On vérifie les conditions 
    \[
        \forall k \in \{1, \dots, d\}, n*p_k \geq 5
    .\]
    C'est la condition de Cochran (1954), il avait testé cas possible en observant l'approximation faites.
    \item Etape 1 : On calcule les effectifs et proportions observées : $ N_{k,n} $ et $ \hat{p}_{k,n} $  
    \item Etape 2 : Calcul de la statistique de test 
    \[
        D = n \sum_{d}^{k=1} \frac{(\hat{p}_{k,n} - p_k)^2}{p_k}
    .\]
    \item Etape 3 : Détermination de la zone de rejet au niveau $ \alpha  $. On lit $ h_\alpha  $ le quantile d'ordre $ 1-\alpha  $ de la loi $ \mathcal{X}^2(d_1) $ 
    \item Etape 4 : Décisions \begin{itemize}
        \item si $ D > h_\alpha  $, on rejette $ H_0 $ (au niveau $ \alpha  $ ). 
        \item Si $ D \leq h_\alpha  $ on conserve $ H_0 $ 
    \end{itemize}
\end{enumerate}

\subsubsection*{Bilan de la méthode}
Aspects positifs : 
\begin{itemize}
    \item \textbf{Fonctionne pour toutes les lois}
    \item Facile à faire
\end{itemize}

Aspects négatifs : 
\begin{itemize}
    \item Problème de consistance. Regrouper les variables par intervalle ruiner l'erreur de seconde espèce.
    \item Asymptotique
    \item Dépendant du choix des intervalles. Ce qui n'est pas canonique.
\end{itemize}

\subsection{Le $ \mathcal{X}^2 $ d'ajustement à une famille paramétrique de loi}
Pratiquement comme avant, pas encore fait en TD, mais copier collé du cours quand même 
\begin{enumerate}
    \item Etape 1 : Soit $ \hat{\theta }_n $ l'estimateur du maximum de vraisemblance de $ \theta  $ (pour $ P_\theta  $ ). On estime \textbf{tous} les paramètres de la loi $ (p_1^{\hat{\theta }_n}, \dots, p_d^{\hat{\theta }_n}) $ 
    \item Etape 2 : On vas tester l'ajustement de $ X_1, \dots, X_n $ à $ P_{\hat{\theta }_n} $ On calcule les fréquences observées $ \hat{p}_{k,n} $.
    \item Etape 3 : Vérification des conditions $ np_k^{\hat{\theta }_n} $ et possible regroupement en classes 
    \item Etape 4 : Calcul de la stat de test $ D $ 
    \item Etape 5 : Zone de rejet : lecture de $ H_\alpha  $ le quantile d'ordre $ 1-\alpha  $ d'une $ \mathcal{X}^2(d-1-M) $ avec $ M $ nombre de paramètre. 
    \item Etape 6 : Décision 
        \begin{itemize}
            \item $ D > h_\alpha  $ on rejette $ H_0 $ 
            \item $ D \leq h_\alpha  $ on conserve $ H_0 $ 
        \end{itemize}
\end{enumerate}


\section{Le test d'homogénéité de Kolmogorov-Smirnov}
\subsection*{Conditions}
\begin{itemize}
    \item Deux échantillons indépendants de variable iid.
\end{itemize}

\subsection*{Hypothèse}
\begin{itemize}
    \item $ H_0 $ : les $ X_i $ et $ Y_i $ ont la même loi, c'est à dire $ F_{X_1} = F_{V_1} $ où $ F_{X_1}, F_{Y_1} $ sont continues.
    \item $ H_1 $ les lois sont différentes
\end{itemize}

\subsection*{Statistique de test}
\[
    \sup _{s \in \mathbb{R}} \left| \frac{1}{n}\sum_{i=1}^{n} \mathbbm{1}_{X_i \leq t} - \frac{1}{n}\sum_{j=1}^{n} \mathbbm{1}_{Y_j \leq t} \right| 
.\]
\subsection*{Zone de Rejet}
\begin{itemize}
    \item Ce test est de taille $ \alpha  $, si on utilise la table de $ h_{n,m} = \sup _{s \in \mathbb{R}} \left| \frac{1}{n}\sum_{i=1}^{n} \mathbbm{1}_{U_i \leq s} - \frac{1}{n}\sum_{j=1}^{n} \mathbbm{1}_{V_j \leq s} \right| $.
    \item Si $ n $ et $ m $ sont trop grands, on utilise le résultat suivant : \\
        Sous $ H_0 $ 
        \[
            \sqrt[]{\frac{nm}{n+m}}h(F_n, G_n) \to ^{\alpha }_{n,m \to +\infty } W_\infty \text{ voir KS asymptotique}
        .\]
        On utilise alors comme zone de rejet $ \sqrt[]{\frac{n+m}{nm}}W_\infty  $ avec $ W_\infty  $ le quantile d'ordre $ 1 - \alpha  $ de $ W_\infty  $.
\end{itemize}

\subsection*{Méthode}


\end{document}