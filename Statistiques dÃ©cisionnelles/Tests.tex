\documentclass{article}
\usepackage[utf8]{inputenc}
\usepackage[a4paper, margin=2.5cm]{geometry}
\usepackage{graphicx}
\usepackage[french]{babel}

\usepackage[default,scale=0.95]{opensans}
\usepackage[T1]{fontenc}
\usepackage{amssymb} %math
\usepackage{amsmath}
\usepackage{amsthm}
\usepackage{systeme}
\usepackage{bbm}
\usepackage{hyperref}
\hypersetup{
    colorlinks=true,
    linkcolor=blue,
    filecolor=magenta,      
    urlcolor=cyan,
    pdftitle={Overleaf Example},
    % pdfpagemode=FullScreen,
    }
\urlstyle{same} %\href{url}{Text}

\theoremstyle{plain}% default
\newtheorem{thm}{Théorème}[section]
\newtheorem{lem}[thm]{Lemme}
\newtheorem{prop}[thm]{Proposition}
\newtheorem*{cor}{Corollaire}
%\newtheorem*{KL}{Klein’s Lemma}

\theoremstyle{definition}
\newtheorem{defn}{Définition}[section]
\newtheorem{exmp}{Exemple}[section]
% \newtheorem{xca}[exmp]{Exercise}

\theoremstyle{remark}
\newtheorem*{rem}{Remarque}
\newtheorem*{note}{Note}
%\newtheorem{case}{Case}



\title{Liste des Tests du cours}
\author{Charles Vin}
\date{2022}

\begin{document}
\maketitle

\section{Test de Kolmogorov-Smirnov}
\subsection*{Conditions}
\begin{enumerate}
    \item Les $ X_i $ semblent provenir d'une loi à fonction de répartition continue. $ \Rightarrow  $ on n'a pas plusieurs fois la même valeur (sauf si celle-ci on était arrondi).\\
    \item Fonctionne $ \forall n $ : même si $ n $ est petit, ce test est pertinent
    \item Si $ n \geq 100 $, on fait un test asymptotique.
\end{enumerate}

\subsection*{Hypothèse}
\begin{itemize}
    \item $ H_0 = $ les $ X_i $ ont pour fdr. $ F_X $ 
    \item $ H_1 = $ les $ X_i $ n'ont pas pour fdr. $ F_X $ 
\end{itemize}

\subsection*{Statistique de test} 
\[
    h(F_n, F) = \max _{1 \leq i \leq n} ( \max ( \left| \frac{i}{n} - F(X_{(i)}) \right| , \left| \frac{i-1}{n}- F(X_{(i)}) \right|  ))
.\]

\subsection*{Zone de Rejet}
\[
    \mathcal{R} = \{h(F_n, F_X) \leq h_{1-\alpha }\}
.\]
avec $ F_n $ fonction de réparation empirique, $ h_{1 -\alpha } $ le quantile à aller chercher dans la table 

\subsection*{Méthode}
Pour trouver la valeur de $ h(F_n, F_X) $ : Faire le grand tableau puis trouver le max. Exemple : 
\begin{table}[!h]
    \centering
    \begin{tabular}{|l|l|l|l|l|l|}
    \hline
        i & 1 & 2 & 3 & 4 & 5 \\ \hline
        $X_{(i)}$ & 0.3 & 0.7 & 0.9 & 1.2 & 1.4 \\ \hline
        $X_{(i)} - 2$ & -1.70 & -1.30 & -1.10 & -0.80 & -0.60 \\ \hline
        $F_0(X_{(i)})$ & 0.04 & 0.10 & 0.14 & 0.21 & 0.27 \\ \hline
        $\frac{i}{n}$ & 0.05 & 0.1 & 0.15 & 0.2 & 0.25 \\ \hline
        $|\frac{i}{n} - F_0(X_{(i)})|$ & 0.01 & 0.00 & 0.01 & 0.01 & 0.02 \\ \hline
        $|\frac{i-1}{n} - F_0(X_{(i)})|$ & 0.04 & 0.05 & 0.04 & 0.06 & 0.07 \\ \hline
    \end{tabular}
    \caption{Ici le max c'est $0.07$ à la dernière case}
\end{table}


\section{Adéquation à une famille d'exponentielle}
\subsection*{Conditions}
\subsection*{Hypothèse}
\subsection*{Statistique de test}
\subsection*{Zone de Rejet}
\subsection*{Méthode}

\section{Adéquation à une loi normale}
\subsection*{Conditions}
\subsection*{Hypothèse}
\subsection*{Statistique de test}
\subsection*{Zone de Rejet}
\subsection*{Méthode}

\section{Le test du $ \mathcal{X}^2 $ d'ajustement}
\subsection*{Conditions}
\begin{enumerate}
    \item Les $ X_i $ sont à valeur dans un ensemble fini (loi discrète)
    \item Test asymptotique : $ \forall k \in \{1, \dots, d\}, np_k^{ref}(1-p_k^{ref}) \geq 5 \Leftrightarrow n \geq 20$ 
\end{enumerate}

\subsection*{Hypothèse}
\begin{align*}
    H_0 &= p = p^{ref} \text{ i.e. } \forall k \in \{1,\dots,d\}, p_k = p_k^{ref} \\
    H_1 &= p \neq p^{ref} \text{ i.e. } \exists k \in \{1, \dots, d\}: p_k \neq p_k^{ref}
\end{align*}
Avec $ p^{ref} $ un vecteur fixé à tester (par exemple pour un lancé de dé $ (\frac{1}{6}, \dots, \frac{1}{6}) $ )

\subsection*{Statistique de test}
\[
    D(\bar{p_n}, p^{ref}) = n \sum_{k=1}^{d}\frac{(\bar{p_{k,n}} - p_k^{ref})^2}{p_k^{ref}} \to ^{\mathcal{L}}_{n \to \infty } \mathcal{X}^2(d-1)
.\]
avec \begin{itemize}
    \item $ N_{k,n} = \sum_{i=1}^{n}\mathbbm{1}_{X_i x_k} $ 
    \item $ \bar{p_{k,n}} = \frac{N_{k,n}}{n} $ les proportions observés
\end{itemize}

\subsection*{Zone de Rejet}
\[
    \mathcal{R} = \{D(\bar{p_n}, p^{ref}) \geq h_\alpha\}
.\]
avec $ h_\alpha  $ le quantile d'ordre $ 1-\alpha  $ de la loi $ \mathcal{X}^2(d-1) $

\subsection*{Méthode}
\begin{enumerate}
    \item Etape 0 : On vérifie les conditions 
    \[
        \forall k \in \{1, \dots, d\}, n*p_k \geq 5
    .\]
    C'est la condition de Cochran (1954), il avait testé cas possible en observant l'approximation faites.
    \item Etape 1 : On calcule les effectifs et proportions observées : $ N_{k,n} $ et $ \hat{p}_{k,n} $  
    \item Etape 2 : Calcul de la statistique de test 
    \[
        D = n \sum_{d}^{k=1} \frac{(\hat{p}_{k,n} - p_k)^2}{p_k}
    .\]
    \item Etape 3 : Détermination de la zone de rejet au niveau $ \alpha  $. On lit $ h_\alpha  $ le quantile d'ordre $ 1-\alpha  $ de la loi $ \mathcal{X}^2(d_1) $ 
    \item Etape 4 : Décisions \begin{itemize}
        \item si $ D > h_\alpha  $, on rejette $ H_0 $ (au niveau $ \alpha  $ ). 
        \item Si $ D \leq h_\alpha  $ on conserve $ H_0 $ 
    \end{itemize}
\end{enumerate}



\end{document}